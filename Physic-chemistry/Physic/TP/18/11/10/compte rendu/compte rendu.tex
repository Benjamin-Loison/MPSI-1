\documentclass{article}
\usepackage{pst-circ}
\begin{document}

	\title{Compte-rendu de travaux pratiques de physique}
	\author{Benjamin Loison, Emilien Goujon et Capucine Gomez (MPSI 1)}
	\date{7 novembre 2018}
	\maketitle

	\section{Etude d'une alimentation stabilisée}

	\subsection{Détermination de $U_{eff}$ à partir de $U_{max}$ en considérant un signal sinusoïdal}

	L'expérimentation nous a permis d'obtenir en suivant le protocole une tension maximale sur l'oscilloscope $U_{max}=2,5$ Volts

	D'après les manuels d'utilisations de l'oscilloscope et du multimètre, on obtient comme approximation: \[ U_{max\ oscillo}=2.5 \pm 3\ \% \]

	\begin{pspicture}[showgrid = false](6, 5)
		\pnodes(0, 0){bottomLeft}(0, 4){topLeft}(5, 0){bottomRight}(5, 4){topRight}
		\vac(bottomLeft)(topLeft){}
		\newdiode(topLeft)(topRight){}
		\wire(bottomLeft)(bottomRight)
		\resistor(topRight)(bottomRight){}
		\newground{135}(bottomRight)
		\wire(topLeft)(1, 5)
		\rput(1.5, 5){\textbf{$y_1$}}
		\wire(topRight)(6, 5)
		\rput(6.5, 5){\textbf{$y_2$}}
	\end{pspicture}

\end{document}