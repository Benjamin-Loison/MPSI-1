\documentclass{article}

\begin{document}

*Friend name*, ceci est mon second document LaTeX (le premier était un pseudo \textit{"Hello World"} :P.

Je me demandais, est-ce que sans la calculatrice on sait dissocier $n$ dans cette inéquation ? Pour avoir par \textbf{exemple}:

\begin{equation}n \ge \frac{ 69 \times 10 ^ 42 }{ \sqrt{2} }\end{equation}

\textbf{Voici la fameuse équation} où j'aimerai obtenir une inégalité comme au-dessus (mais sans utiliser la calculatrice pour simplifier quoi que ce soit):

\begin{equation}\left( \frac{ 1 - \sqrt{2} }{ 1 + \sqrt{2} } \right) ^ {2 ^ n} \ge \frac{2}{10 ^ 5(3 + 2 \sqrt{2})}\end{equation}

Personnellement je l'aime bien ce vrai premier fichier en TeX :D Et vous, qu'en pensez vous ?

\end{document}
